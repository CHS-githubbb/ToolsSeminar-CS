\documentclass[sigconf]{acmart}
\usepackage{algorithmic}
\usepackage{algorithm}

\title{A Short English Academic Paper}

\author{Your Name}
\affiliation{%
  \institution{Your Affiliation}
}
\email{your.name@email.com}


\begin{document}

\begin{abstract}
This article illustrates the common usage of \LaTeX{} commands for English academic papers.
ACM conference format is used to style the document.
\end{abstract}

\maketitle

\section{Introduction}
This is the Introduction part.
Be careful of the indentation of English articles.
The sentence following the section title has no indentation.

See! There is an indentation from the second paragraph.

\section{Related Works}
Once you mention others' methods, conclusions, experiments, etc., you should cite their works.
For example, AI~\cite{sivathanu_astra_2019,hua_boosting_2019,deng_tie_2019,frankle_lottery_2019} and graph~\cite{ma_neugraph_2019,besta_slim_2019,dhulipala_low-latency_2019,dong_network_2019} are two hot topics nowadays.

\section{Background}
Here gives the background.

\section{Methodologies}
Commonly, we use \emph{Italic} fonts instead of \textbf{bold} fonts to emphasize something in English articles.

\subsection{Methodology 1}
We do not directly list source codes in the paper, but use pseudocode like Alg.~\ref{alg:count_ones} to demonstrate our algorithms.
\begin{algorithm}[htbp]
\caption{Count \# of ones}
\label{alg:count_ones}
\small
\begin{algorithmic}[1]  
\REQUIRE array $a$, size $n$
\STATE Initialize $cnt = 0$
\FOR{$i = 0$ \textbf{to} $n-1$}
\IF{$a[i]$ is $1$}
\STATE $cnt \gets cnt + 1$
\ENDIF
\ENDFOR
\RETURN $cnt$
\end{algorithmic}
\end{algorithm}

\subsection{Methodology 2}
More methodologies here.

\section{Experiments}
This section firstly introduces the experimental settings and then presents the experimental results.

\subsection{Experimental Settings}
We use the \verb'acmart' \LaTeX{} template to format this article.

\subsection{Experimental Results}
Some experimental results are shown in Table~\ref{tab:ex}.
\begin{table}[htbp]
\caption{A table example}
\label{tab:ex}
\begin{tabular}{|c|c|c|}\hline
Col 1 & Col 2 & Col 3\\\hline
Line 1 & & \\\hline
\end{tabular}
\end{table}

\section{Conclusions}
This article gives a basic structure of English academic paper and the usage of \verb'acmart' template.

\section*{Acknowledgement}
This section can be omitted. But you should make sure the contents have filled the whole pages.

\bibliographystyle{unsrt}
\bibliography{exe_en.bib}

\end{document}